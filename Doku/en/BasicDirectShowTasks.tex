\subsection{Displaying a Filter's Property Pages}
\textbf{Displaying a Filter's Property Pages}

A property page is one way for a filter to support properties that the user can set. This article describes how to display a filter's property pages in an application. For more information about property pages, see the Platform SDK documentation.

\begin{description}
	\item[Note]  Although many of the filters provided with DirectShow support property pages, they are intended for debugging purposes, and are not recommended for application use. In most cases the equivalent functionality is provided through a custom interface on the filter. An application should control these filters programmatically, rather than expose their property pages to users.
\end{description}

Filters with property pages expose the ISpecifyPropertyPages interface. To determine whether a filter defines a property page, query the filter for this interface using QueryInterface.
If you directly created an instance of a filter (by calling CoCreateInstance), you already have a pointer to the filter. If not, you can enumerate the filters in the graph, using the IFilterGraph::EnumFilters method. For details, see Enumerating Objects in a Filter Graph.

Once you have the ISpecifyPropertyPages interface pointer, retrieve the filter's property pages by calling the ISpecifyPropertyPages::GetPages method. This method fills a counted array of globally unique identifiers (GUIDs) with the class identifier (CLSID) of each property page. A counted array is defined by a CAUUID structure, which you must allocate but do not have to initialize. The GetPages method allocates the array, which is contained in the pElems member of the CAUUID structure. When you are done, free the array by calling the CoTaskMemFree function.

The OleCreatePropertyFrame function provides a simple way to display the property pages inside a modal dialog box.

\begin{verbatim}
C++

IBaseFilter *pFilter;
/* Obtain the filter's IBaseFilter interface. (Not shown) */
ISpecifyPropertyPages *pProp;
HRESULT hr = pFilter->QueryInterface(IID_ISpecifyPropertyPages, (void **)&pProp);
if (SUCCEEDED(hr)) 
{
	// Get the filter's name and IUnknown pointer.
	FILTER_INFO FilterInfo;
	hr = pFilter->QueryFilterInfo(&FilterInfo); 
	IUnknown *pFilterUnk;
	pFilter->QueryInterface(IID_IUnknown, (void **)&pFilterUnk);
	
	// Show the page. 
	CAUUID caGUID;
	pProp->GetPages(&caGUID);
	pProp->Release();
	OleCreatePropertyFrame(
	hWnd,                   // Parent window
	0, 0,                   // Reserved
	FilterInfo.achName,     // Caption for the dialog box
	1,                      // Number of objects (just the filter)
	&pFilterUnk,            // Array of object pointers. 
	caGUID.cElems,          // Number of property pages
	caGUID.pElems,          // Array of property page CLSIDs
	0,                      // Locale identifier
	0, NULL                 // Reserved
	);
	
	// Clean up.
	pFilterUnk->Release();
	FilterInfo.pGraph->Release(); 
	CoTaskMemFree(caGUID.pElems);
}

\end{verbatim}