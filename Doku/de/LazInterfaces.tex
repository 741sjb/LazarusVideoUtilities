\subsection{Interfaces}
\subsubsection{IMoniker}
Instanzen haben in COM keine GUIDs, sondern Moniker. Moniker werden intern selbst in Objekten (Moniker-Objekten mit der Schnittstelle IMoniker) gespeichert. Ein COM-Server kann einem COM-Client ein Moniker-Objekt �bergeben, um dem Client zu erm�glichen, zu einem sp�teren Zeitpunkt wieder eine Verbindung zu dieser Instanz aufzunehmen, nachdem die Objektreferenz zwischenzeitlich gel�scht wurde.
\paragraph{Allgemein}
Erm�glicht es Ihnen, ein Moniker-Objekt zu verwenden, das Informationen enth�lt, die ein COM-Objekt eindeutig identifizieren. Ein Objekt, das einen Zeiger auf die IMoniker-Schnittstelle des Monikerobjekts hat, kann den identifizierten Objekt lokalisieren, aktivieren und erhalten, ohne irgendwelche anderen spezifischen Informationen dar�ber zu haben, wo sich das Objekt tats�chlich in einem verteilten System befindet.
Moniker werden als Grundlage f�r die Verkn�pfung in COM verwendet. Ein verkn�pftes Objekt enth�lt einen Moniker, der seine Quelle identifiziert. Wenn der Benutzer das verkn�pfte Objekt aktiviert, um es zu bearbeiten, ist der Moniker gebunden, dies l�dt die Linkquelle in den Speicher.
\paragraph{Besonderheiten}
Wie ein Pfad zu einer Datei in einem Dateisystem, enth�lt ein Moniker Informationen, die es erm�glichen, ein COM-Objekt zu lokalisieren und zu aktivieren. Moniker k�nnen jede Art von COM-Objekt aus einem in einer Datei gespeicherten Dokumentobjekt zu einer Auswahl innerhalb eines eingebetteten Objekts identifizieren. COM bietet eine Reihe von Moniker-Klassen, mit denen Sie Moniker-Objekte erstellen k�nnen, die die am h�ufigsten im System gefundenen Objekte identifizieren. Zum Beispiel k�nnte es ein Objekt sein, das einen Bereich von Zellen in einer Kalkulationstabelle darstellt, die selbst in einem Textdokument eingebettet ist, das in einer Datei gespeichert ist. In einem verteilten System w�rde der Moniker dieses Objekts den Ort des Objektsystems, den physischen Standort der Datei auf diesem System, die Speicherung des eingebetteten Objekts innerhalb dieser Datei und schlie�lich die Position des Zellenbereichs innerhalb des eingebetteten Objekts identifizieren.

Ein Moniker-Objekt unterst�tzt die IMoniker-Schnittstelle, die aus der IPersistStream-Schnittstelle abgeleitet wird und ein einzelnes Objekt im System eindeutig identifiziert. Nachdem ein Objekt, das einen Moniker bereitstellt, das Moniker-Objekt erstellt hat, kann diese Information innerhalb dieses Objekts nicht ge�ndert werden. Wenn der Moniker-Provider die Informationen �ndert, kann er dies nur durch die Erstellung eines neuen Moniker-Objekts tun, das dann das betreffende Objekt eindeutig identifizieren w�rde.

Moniker haben folgende zwei wichtige F�higkeiten:
\begin{itemize}
	\item Moniker k�nnen in einem persistenten Speicher gespeichert werden. Wenn ein Moniker wieder in den Speicher geladen wird, identifiziert er immer noch dasselbe Objekt.
	\item Monikers unterst�tzen eine Operation namens Bindung, die der Prozess des Lokalisierens des vom Moniker benannten Objekts ist, das Objekt wird ind den Speicher geladen, wenn es nicht bereits aktiv ist und ein Zeiger zur�chgegeben der auf die angeforderte Schnittstelle des Objekt zeigt.
\end{itemize}

\paragraph{Methoden}
Beschreibung von Microsoft\footnote{Quelle: https://msdn.microsoft.com/en-us/library/windows/desktop/ms679705(v=vs.85).aspx}

\begin{tabular}{|r|l|}
	\hline 
	\textbf{Method}&\textbf{Description}\\ 
	\hline
	BindToObject&Bind to the specified object.\\ 
	\hline
	BindToStorage&Binds to the storage for the specified object.\\ 
	\hline
	CommonPrefixWith&Creates a new moniker based on the prefix that this moniker has in common with the specified moniker.\\ 
	\hline
	ComposeWith&Creates a new composite moniker by combining the current moniker with the specified moniker.\\ 
	\hline
	Enum&Retrieves a pointer to an enumerator for the components of a composite moniker.\\ 
	\hline
	GetDisplayName&Retrieves the display name for the moniker.\\ 
	\hline
	GetTimeOfLastChange&Retrieves the time at which the object identified by this moniker was last changed.\\ 
	\hline
	Hash&Creates a hash value using the internal state of the moniker.\\ 
	\hline
	Inverse&Creates a moniker that is the inverse of this moniker.\\ 
	\hline
	IsEqual&Determines whether this moniker is identical to the specified moniker.\\ 
	\hline
	IsRunning&Determines whether the object identified by this moniker is currently loaded and running.\\ 
	\hline
	IsSystemMoniker&Determines whether this moniker is one of the system-provided moniker classes.\\ 
	\hline
	ParseDisplayName&Converts a display name into a moniker.\\ 
	\hline
	Reduce&Reduces a moniker to its simplest form.\\ 
	\hline
	RelativePathTo&Creates a relative moniker between this moniker and the specified moniker.\\ 
	\hline
\end{tabular} 


\begin{lstlisting}
// part of ActiveX
PIMoniker = ^IMoniker;
IMoniker = Interface (IPersistStream)
  ['{0000000f-0000-0000-C000-000000000046}']
  Function BindToObject (const pbc:IBindCtx;const mktoleft:IMoniker; const RiidResult:TIID;Out vresult):HResult;StdCall;
  Function BindToStorage(Const Pbc:IBindCtx;
  				Const mktoLeft:IMoniker;
  				const Riid:TIID;
  				Out vobj):HResult; StdCall;
  Function Reduce (const pbc:IBindCtx; 
               dwReduceHowFar:DWord; 
               mktoLeft: PIMoniker; 
               Out mkReduced:IMoniker):HResult; StdCall;
  Function ComposeWith(Const MkRight:IMoniker;
               fOnlyIfNotGeneric:BOOL; 
               OUT mkComposite:IMoniker):HResult; StdCall;
  Function Enum(fForward:Bool;
               Out enumMoniker:IEnumMoniker):HResult;StdCall;
  Function IsEqual(Const mkOtherMoniker:IMoniker):HResult;StdCall;
  Function Hash   (Out dwHash:Dword):HResult;StdCall;
  Function IsRunning(Const bc:IBindCtx;
               Const MkToLeft:IMoniker;
               Const mknewlyRunning:IMoniker):HResult;StdCall;
  Function GetTimeOfLastChange(Const bc:IBindCtx;
               Const mkToLeft:IMoniker; 
               out ft : FileTime):HResult; StdCall;
  Function Inverse(out mk : IMoniker):HResult; StdCall;
  Function CommonPrefixWith (Const mkOther:IMoniker):HResult; StdCall;
  Function RelativePathTo(Const mkother:IMoniker; 
               Out mkRelPath : IMoniker):HResult;StdCall;
  Function GetDisplayName(Const bc:IBindCtx;
               const mktoleft:IMoniker;
               Out szDisplayName: pOleStr):HResult; StdCall;
  Function ParseDisplayName(Const bc:IBindCtx;
               Const mkToLeft:IMoniker;
               szDisplayName:POleStr;
               out cheaten:ULong;
               out mkOut:IMoniker):HResult; StdCall;
  Function IsSystemMoniker(Out dwMkSys:DWord):HResult;StdCall;
End;
\end{lstlisting}
